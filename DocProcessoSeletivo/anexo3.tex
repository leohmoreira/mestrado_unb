\documentclass[a4paper,12pt]{article}
\usepackage[brazilian]{babel}
\usepackage[utf8]{inputenc}
\usepackage[T1]{fontenc}
\usepackage{indentfirst}
\usepackage{graphicx}
\usepackage{fancyhdr}
\usepackage{setspace}

% include --------------------------------------------------------------------------------------------------------------------
% \pagestyle{fancy}
\lhead{
  \begin{minipage}[c]{0.1\linewidth}
   \includegraphics[scale=0.5]{images/logoUnB_mini.png}
  \end{minipage}
}
\rhead{
  \hfill
  \begin{minipage}[c]{0.9\linewidth}
   \fontsize{6}{8}\selectfont
   Universidade de Brasília \\
   Instituto de Exatas - IE \\
   Programa de Pós-gradução em Informática \\
   Edital n. 02/2015
  \end{minipage}
}
\cfoot{
\fontsize{6}{8}\selectfont
\textbf{Universidade de Brasília – Campus Universitário Darcy Ribeiro – CEP 70910-900 – Brasília/DF - Brasil}\\
Instituto de Ciências Exatas - Departamento de Ciência da Computação – Caixa postal: 4466\\
Telefone/Fax: (61) 3107-3661\\
www.cic.unb.br / ppgi.unb.br/
}

% include --------------------------------------------------------------------------------------------------------------------

\begin{document}
\onehalfspacing
%comandos-----------------------------------------------------------------------------------------------------------------
\newcommand{\linhaDupla}{\newline \newline}

%comandos-----------------------------------------------------------------------------------------------------------------

% \begin{figure}
%  \includegraphics[width=0.8\textwidth]{images/logoUnB.png}
% \end{figure}
% 
% \textbf{PROGRAMA DE PÓS-GRADUAÇÃO EM INFORMÁTICA}
% \linhaDupla
% \textbf{Nome:} Leonardo Henrique Moreira
% \linhaDupla
% \textbf{CPF:} 294.490.948-70
% \linhaDupla
% \textbf{Proposta de Projeto de Pesquisa}
% \linhaDupla
% \textbf{Título:}
% \linhaDupla
% \textbf{Linha de Pesquisa:}
% \linhaDupla
% \indent\textbf{Área de Pesquisa:}
% 
% \newpage
\section{Introdução}
A capacidade do homem em perceber, interpretar e interagir com o ambiente no qual está inserido depende de um elemento essencial, a informação.
O estado atual da tecnologia, aliado à sua capacidade de cobertura global e em tempo real, aumenta a facilidade de acesso à informação e dessa forma as pessoas 
tem sido apresentadas a um volume de dados cada vez maior. Apesar de muitos destes dados serem relevantes, existem também aqueles que não agregam valor.\\
\indent De maneira especial, em um cenário militar, essa grande quantidade de informação pode prejudicar o processo decisório do 
comandante pois este volume, aliado à velocidade com a qual elas são apresentadas, pode dificultar a definição de qual tratamento cada uma deve receber.
A sobrecarga no decisor pode causar tanto atenção excessiva a fatos pouco relevantes quanto distração sobre pontos importantes.\\
\indent A condição ideal para o decisor, em qualquer escalão, é que exista um equilíbrio entre o fluxo de informação e o processo de interpretação dos dados.
Entretanto, apesar da alta taxa na qual dados tem sido coletados oriundos do grande número de sensores e sistemas utilizados para os mais diversos fins, 
verifica-se que a presença de ferramentas e algoritmos capazes de tratar esse volume de informação não existem [Savas et al 2014].\\
\indent Face a este problema, esta proposta de estudo visa basear-se nas áreas de inteligência artifical, teoria dos jogos e suporte à decisão em Sistemas 
de Comando e Controle (C2) tendo como foco principal o uso destas teorias para ordenar as informações de tal forma que o valor final agregado ao decisor seja 
maximizado.\\
\indent A construção deste modelo é importância ao Exército Brasileiro porque através dele é possível auxiliar o processo de tomada de 
decisão devido ao fato de evitar ou diminuir a sobrecarga de informação nos decisores. Outro aspecto que aumenta a 
contribuição desta proposta de trabalho é o contínuo emprego das Forças Armadas, em especial o Exército, em operações 
de garantia da lei e da ordem e de defesa interna durante as quais a cobertura dos veículos de jornalismo garante a 
comunicação da informação praticamente em tempo real.

\section{Justificativa}
De acordo com a doutrina de Comando e Controle relaciona a qualidade de informação por meio de sete critérios: precisão, oportunidade, facilidade de uso,
suficiência, brevidade e segurança. O segundo item merece atenção especial por relacionar o instante da coleta da informação com o momento até o qual ela é 
útil para o processo de toma de decisão.\\
\indent Segundo [Savas et al 2014], não existem ferramentas e algoritmos capazes de tratar o volume de informação, principalmente nos critérios supracitados. Aliando à
essa condição a busca pela superioridade de informação, isto é, a habilidade de coletar e processar dados, torna-se evidente a necessidade de
contribuições científicas e técnicas nesta área.

\section{Objetivos}
\indent O objetivo do estudo é criar um método capaz de realizar o ordenamento da lista de informações presente no sistema de 
Comando e Controle utilizado. A motivação para tal tratamento é oferecer ao decisor a possibilidade de avaliar, de maneira 
prioritária, as informações mais importantes ou mais urgentes, uma vez que tratá-las na ordem de chegada pode não ser 
eficiente.\\
\indent Como o exercício de C2 está presente nos diversos níveis hierárquicos das operações, a valorização de cada item 
da lista depende da sua função de combate (movimento e manobra, inteligência, logística, fogos, proteção), do escalão de 
seu decisor, da urgência no seu atendimento e principalmente do seu instante de chegada. Outra característica importante 
deste método deve ter é ser capaz de ser implementado e integrado à Família de Aplicativos de Comando e Controle da 
Força Terrestre.\\
\section{Revisão da literatura}
O Modelo Dinâmico de Cognição Situada (\textit{Dynamic Model of Situated Cognition - DMSC}) foi introduzido pela primeira vez em 2003 e rapidamente ganhou 
aceitacao na comunidade do DoD como um modelo descritivo de processamento e fluxo de dados e informacoes no campo de batalha [Phillips 2008].\\
\indent Este modelo demonstra a relação entre os sistemas tecnológicos e processos perceptivos e cognitivos humanos, fornecendo um quadro eficaz no qual se pode
ver o fluxo de dados e informações em um sistema complexo. Apesar de não ser um modelo analítico, o DMSC fornece um meio de ver o problema do fluxo de dados e 
informações de uma forma que seja acessível não somente para os analistas como também para os decisores. O DMSC é a base para a construção do modelo matemático
 para qualificação da informação.\\
\indent Além desse trabalho, outras pesquisas relatam a análise de informação no cenário de Comando e Controle. Entre elas destacam-se o uso de teoria dos 
jogos para predição de ameaças e análise de situação [Brynielsson et al 2004] e os desafios para sistema de apoio à decisão nesse cenário, principalmente em 
relação ao impacto do atraso no tratamento de informação.\\


\section{Metodologia}
\indent Para atingir o objetivo estabelicido, será primeiramente conduzida uma pesquisa bibliográfica relativa a Comando e Controle, buscando mapear as ligações 
com as áreas de inteligência artificial, teoria dos jogos e suporte à decisão. A partir daí, será feito um estudo mais detalhado do processo de qualificação da
informação com objetivo de produzir um modelo matemático que considere os critérios, principalmente o da oportunidade.\\
\indent Com a definição desse modelo, serão estudados cenários de uso e serão realizadas simulações verificando qual o ganho do ordenamento das informações para
 o processo decisório.
\section{Plano de trabalho}
O curso será realizado através das seguintes atividades (não necessariamente sequenciais):
\begin{itemize}
 \item cursar disciplinas necessárias;
 \item executar pesquisa bibliográfica sobre XXX;
 \item defender qualificação com proposta de pesquisa específica;
 \item elaborar a dissertação;
 \item defender a dissertação;
\end{itemize}

\section{Cronograma}
\section{Referência bibliográficas}

\begin{thebibliography}{1}

\bibitem{Savas 2014}
O. Savasa, Y. Sadguyu, J. Deng, J. Li, \emph{Tactical big data analytics: challenges, use cases and solutions.} In \textit{ACM SIGMETRICS Performance Evaluation
 Review}, volume 41, pages 86-89.
\bibitem{Liu:2011}
K. J. R. Liu and B. Wang, \emph{Cognitive Radio Networking and Security}, Cambridge University Press, 2011.

\bibitem{Chen:2008a}
R. Chen, J. Park and J. H. Reede, \emph{Defense against primary user emulation attacks in cognitive radio networks}, 
IEEE Journal on Selected Areas in Communications, v. 26, pp. 25--37, 2008.

\end{thebibliography}

\end{document}
