\documentclass[a4paper,12pt]{article}

\usepackage[brazilian]{babel}
\usepackage[utf8]{inputenc}
\usepackage[T1]{fontenc}
\usepackage{indentfirst}
\usepackage{graphicx}

\pdfinfo{%
   /Title    (Proposta de Mestrado)
   /Author   (Leonardo Moreira)
   /Creator  (Leonardo Moreira)
   /Producer ()
   /Subject  ()
   /Keywords ()
 }

\usepackage{fancyhdr}
 
% \pagestyle{fancy}
% \fancyhf{}
% \rhead{Proposta de Projeto de Pesquisa - Mestrado}
% \lhead{\includegraphics[width=0.3\textwidth]{images/logoUnB.png}}
% \rfoot{Página \thepage}
% \lfoot{Leonardo Henrique Moreira}

\pagestyle{fancy}
\fancyhf{}
% \rhead{Proposta de Projeto de Pesquisa - Mestrado}
\lhead{
  \fontsize{6}{8}\selectfont
  \includegraphics[scale=0.5]{images/logoUnB_mini.png}
  Universidade de Brasília \\
  Instituto de Exatas - IE \\
  Programa de Pós-gradução em Informática \\
  Edital n. 02/2015
}
\cfoot{
Universidade de Brasília – Campus Universitário Darcy Ribeiro – CEP 70910-900 – Brasília/DF - Brasil
Instituto de Ciências Exatas - Departamento de Ciência da Computação – Caixa postal: 4466
Telefone/Fax: (61) 3107-3661
www.cic.unb.br / ppgi.unb.br/
}

\begin{document}
%comandos-----------------------------------------------------------------------------------------------------------------
\newcommand{\linhaDupla}{\newline \newline}

%comandos-----------------------------------------------------------------------------------------------------------------

\begin{figure}
\thispagestyle{empty}
 \includegraphics[width=\textwidth]{images/logoUnB.png}
\end{figure}
\textbf{PROGRAMA DE PÓS-GRADUAÇÃO EM INFORMÁTICA}
\linhaDupla
\textbf{Nome:} Leonardo Henrique Moreira
\linhaDupla
\textbf{CPF:} 294.490.948-70
\linhaDupla
\textbf{Proposta de Projeto de Pesquisa}
\linhaDupla
\textbf{Título:}
\linhaDupla
\textbf{Linha de Pesquisa:}
\newline
\indent\textbf{Área de Pesquisa:}

\newpage

\section{Introdução}
Com a grande facilidade de acesso à informação atualmente, as pessoas acabam sendo apresentadas a um grande volume de dados. 
Muitos destes dados são relevantes porém existem aqueles que não agregam valor. 
De maneira especial, em um cenário militar, uma grande quantidade de informações pode prejudicar o processo decisório do 
comandante pois este volume aliado à velocidade com a qual elas são apresentadas podem dificultar a definição de qual 
tratamento cada uma deve receber.
Face a este problema, esta proposta de estudo baseia-se nas áreas de inteligência artifical e suporte à decisão em Sistemas 
de Comando e Controle (C2) tendo como foco principal o uso destas teorias para ordenar as informações de tal forma que o 
valor final agregado ao decisor seja maximizado.
O objetivo do estudo é criar um método capaz de realizar o ordenamento da lista de informações presente no sistema de 
Comando e Controle utilizado. A motivação para tal tratamento é oferecer ao decisor a possibilidade de avaliar, de maneira 
prioritária, as informações mais importantes ou mais urgentes, uma vez que tratá-las na ordem de chegada pode não ser 
eficiente. Como o exercício de C2 está presente nos diversos níveis hierárquicos das operações, a valorização de cada item 
da lista depende da sua função de combate (movimento e manobra, inteligência, logística, fogos, proteção), do escalão de seu 
decisor, da urgência no seu atendimento e principalmente do seu instante de chegada. Outra característica importante deste 
método deve ter é ser capaz de ser implementado e integrado à Família de Aplicativos de Comando e Controle da Força 
Terrestre.
A construção deste modelo é de importância ao Exército porque através dele é possível auxiliar o processo de tomada de 
decisão devido ao fato de evitar ou diminuir a sobrecarga de informação nos decisores. Outro aspecto que aumenta a 
contribuição desta proposta de trabalho é o contínuo emprego das Forças Armadas, em especial o Exército, em operações de 
garantia da lei e da ordem e de defesa interna durante as quais a cobertura dos veículos de jornalismo garante a comunicação
da informação praticamente em tempo real.

\section{Justificativa}
\section{Objetivos}
\section{Revisão da literatura}
\section{Metodologia}
\section{Plano de trabalho}
\section{Cronograma}
\section{Referência bibliográficas}
\end{document}
